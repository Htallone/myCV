%%%% 参考了https://www.wondercv.com/的模板

\documentclass[11pt]{article}

% disable indent globally
\setlength{\parindent}{0pt}
% some general improvements, defines the XeTeX logo
\usepackage{xltxtra}
% use hyperlink for email and url
\usepackage{hyperref}
\hypersetup{hidelinks}
\usepackage{url}
\urlstyle{tt}

\usepackage{xcolor}
%%%% 统一一种颜色,偏蓝色,用于section下划线和fontawesome,这颜色是从一个北航Logo上取的
\definecolor{CVBlue}{RGB}{23,110,191}

%%% \widthof[]{} 用于特殊对齐是用到
\usepackage{calc}


% loading fonts
\usepackage{fontspec}
\usepackage{xeCJK}
\CJKsetecglue{} %% 取消中文与数字之间间隙

%%%%% 字体需要自己下载安装,注意版权问题,这两种字体应该比较好看,英文Helvetica,中文方正兰亭黑,也是有多种版本,自己试试哪些好看。参考了https://www.wondercv.com/的模板
%%%%% windows系统好像需要先安装字体,之后下面语句就够了
% Main document font
% \setmainfont[
%   BoldFont = HelveticaNeueLTPro-Md.otf ,
% ]{HelveticaNeueLTPro-Roman.otf}
% 
% \setCJKmainfont[
% BoldFont=Pro_GB18030 DemiBold.otf,
% ]{Pro_GB18030.otf}

%%%%% 字体需要自己下载安装,注意版权问题
%%%%% linux系统只需要字体路径就行了,如下
% % Main document font
\setmainfont[
    Path = Font/,
  Extension = .otf ,
  BoldFont = HelveticaNeueLTPro-Md.otf ,
]{HelveticaNeueLTPro-Roman.otf}

\setCJKmainfont[
Path = Font/,
  Extension = .otf ,
BoldFont=Pro_GB18030 DemiBold.otf,
]{Pro_GB18030.otf}

%%%%% 定义更漂亮的“C++”,参考https://tex.stackexchange.com/questions/4302/prettiest-way-to-typeset-c-cplusplus 
%%%%% 貌似跟具体字体大小有关,需要调下参数,我测试感觉下面的比较好看
\usepackage{relsize}
\usepackage{xspace}
\protected\def\Cpp{{C\nolinebreak[4]\hspace{-.05em}\raisebox{.28ex}{\relsize{-1}++}}\xspace} 

% use fontawesome
\usepackage{fontawesome}
\newfontfamily{\FA}{[FontAwesome.otf]}

\usepackage[
	a4paper,
	left=1.2cm,
	right=1.2cm,
	top=1.5cm,
	bottom=1cm,
	nohead
]{geometry}

\renewcommand{\baselinestretch}{1.2} %定义行间距1.2

\usepackage{titlesec}
\usepackage{enumitem}
\setlist{noitemsep} % removes spacing from items but leaves space around the whole list
%\setlist{nosep} % removes all vertical spacing within and around the list
\setlist[itemize]{topsep=0.25em, leftmargin=*}
\setlist[enumerate]{topsep=0.25em, leftmargin=*}



\titleformat{\section}         % Customise the \section command 
  {\large\bfseries\raggedright} % Make the \section headers large (\Large),
                               % small capitals (\scshape) and left aligned (\raggedright)
  {}{0em}                      % Can be used to give a prefix to all sections, like 'Section ...'
  {}                           % Can be used to insert code before the heading
  [{\color{CVBlue}\titlerule}]                 % Inserts a horizontal line after the heading
\titlespacing*{\section}{0cm}{*1.6}{*1.2}



\begin{document}
\pagenumbering{gobble} % suppress displaying page number

\centerline{\LARGE\bfseries{高富帅}}

\centerline{\normalsize{应届博士生,研究方向:飞行力学、制导与控制、弹道优化、拦截仿真}}

\centerline{\normalsize{\faPhone\ 158-8888-8888 \quad \faEnvelopeO\ \href{mailto:Htallone@buaa.edu.cn}{Htallone@buaa.edu.cn}}}
%   \vspace{1.5ex}
 
\section{\makebox[\widthof{\faGraduationCap}][c]{\color{CVBlue}\faGraduationCap}\  教育背景}

\textbf{北京航空航天大学} \hfill 2011年 -- \makebox[\widthof{2011年}][s]{现在}

博士研究生\quad 飞行器设计,硕博连读,预计2019年3月毕业

\textbf{北京航空航天大学} \hfill 2008年 -- 2011年

理学学士第二学位\quad 应用数学

\textbf{北京航空航天大学} \hfill 2007年 -- 2011年

工学学士学位\quad 飞行器设计与工程(航天工程)

\section{\makebox[\widthof{\faGraduationCap}][c]{\color{CVBlue}\faUsers}\ 项目经历}

\textbf{黑科技公司}\  \hfill 2015年3月 -- 2015年5月

实习\quad 经理: 高富帅 \hfill 北京

xxx后端开发
\begin{itemize}
  \item 实现了 xxx 特性
  \item 后台资源占用率减少8\%
  \item xxx
\end{itemize}

\textbf{六自由度拦截弹攻防对抗模型\Cpp}\  \hfill 2015年3月 -- 2015年5月

实习\quad 经理: 高富帅  \hfill 武汉

xxx后端开发\Cpp
\begin{itemize}
  \item 实现了 xxx 特性
  \item 后台资源占用率减少8\%
  \item xxx
\end{itemize}

\textbf{高富帅是真的吗?}\  \hfill 2015年3月 -- 2015年5月

自由探索  \hfill 瓦房店

高富帅需要简历?哪个高富帅学这个专业?怕不是个傻子吧!
\begin{itemize}
  \item 实现了 xxx 特性,哪个高富帅学这个专业?怕不是个傻子吧!哪个高富帅学这个专业?怕不是个傻子吧!哪个高富帅学这个专业?怕不是个傻子吧!
  \item 后台资源占用率减少8\%
\end{itemize}


\section{\makebox[\widthof{\faGraduationCap}][c]{\color{CVBlue}\faCogs}\ IT 技能}
% increase linespacing [parsep=0.5ex]
\begin{itemize}[parsep=0.5ex]
  \item 编程语言: C == Python > \Cpp > Java
  \item 平台: Linux
  \item 开发: 英语六级,博士期间阅读了大量专业英文文献、开源项目英文文档等。
\end{itemize}

\section{\makebox[\widthof{\faGraduationCap}][c]{\color{CVBlue}\faHeart}\ 获奖情况}
第一名, xxx 比赛 \hfill 2013年6月

其他奖项 \hfill 2015

\section{\makebox[\widthof{\faGraduationCap}][c]{\color{CVBlue}\faInfo}\ 其他}
% increase linespacing [parsep=0.5ex]
\begin{itemize}[parsep=0.5ex]
  \item 技术博客: http://blog.yours.me
  \item GitHub: https://github.com/username
  \item 语言: 英语 - 熟练(TOEFL xxx)
\end{itemize}

%%%% 如果多页简历,可以手动在适当位置插入 \newpage 或者 \clearpage 开始新一页

\end{document}
